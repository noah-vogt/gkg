% standard
\documentclass[a4paper,12pt]{article}
\usepackage[utf8]{inputenc}
\usepackage[ngerman]{babel}

% geometry
\usepackage{geometry}
\geometry{ headsep=20pt,
headheight=20pt,
left=21mm,
top=15mm,
right=21mm,
bottom=15mm,
footskip=20pt,
includeheadfoot}

% header and footer
\usepackage{datetime}
\newdateformat{dmy}{%
\THEDAY.~\monthname[\THEMONTH] \THEYEAR}
\usepackage{fancyhdr}
\pagestyle{fancy}
\lhead{Noah Vogt \& Simon Hammer}
\chead{}
\rhead{\dmy\today}
\lfoot{}
\cfoot{Gymnasium Kirschgarten}
\rfoot{Seite \thepage}
\renewcommand{\footrulewidth}{.4pt}

% fix figure positioning
\usepackage{float}

% larger inner table margin
\renewcommand{\arraystretch}{1.4}

% no paragraph indent
\setlength{\parindent}{0em}

% graphics package
\usepackage{graphicx}

\usepackage{multicol}

% use sans serif font
\usepackage{tgheros}
\usepackage{mathptmx}

% don't even ask what this is for, I have no idea (noah)
\usepackage{bm} %italic \bm{\mathit{•}}
\usepackage[hang]{footmisc}
\usepackage{siunitx}
\usepackage[font={small,it}]{caption}
\sisetup{locale = DE, per-mode = fraction, separate-uncertainty,   exponent-to-prefix, prefixes-as-symbols = false, scientific-notation=false
}
\newcommand{\ns}[4]{(\num[scientific-notation=false]{#1}\pm\num[scientific-notation=false]{#2})\cdot\num[]{e#3}\si{#4}}

% show isbn in bibliography
\usepackage{natbib}

\begin{document}

\begin{titlepage}

\vspace*{1cm}
	\centering
	
	{\scshape\Large Protokolle Praktikum Physik 3cg \par}
	\vspace{0.5cm}
	{\huge\bfseries die experimentelle Bestimmung der Kapazität eines unbekannten Kondensators\par}
	\vspace{0.5cm}
	{\Large Noah Vogt \& Simon Hammer\par}
	\vspace{17cm}

	{\large Durchgeführt am 27. Oktober 2020\par}
	
\end{titlepage}

\tableofcontents
\pagebreak

\section{Versuchsziel}
Ziel ist es die \textit{Kapazität} $L_f$ eines Kondensators mittels eines \textit{Experiments} so genau wie möglich zu bestimmen, indem ein Kondensator mit bekannter Kapazität genommen wird und mit einer bestimmten Anzahl Volt aufgeladen wird. Durch ein Zusammenschliessen der beiden Kondensatoren wird der zweite auch mitaufgeladen, sodass beide Kondensatoren die gleiche Spannung haben (\textit{Spannungsausgleich}). Nun kann mithilfe einiger Formeln die gesuchte Kapazität des unbekannten Kondensators besimmt werden. Vorausschtlich wird der Messwert unter dem realen Wert liegen aufgrund systematischer Fehler.

\section{Physikalischer Hintergrund}

Die elektrische Kapazität (Formelzeichen C, von lateinisch capacitas ‚Fassungsvermögen‘; Adjektiv kapazitiv) ist eine physikalische Größe aus dem Bereich der Elektrostatik, Elektronik und Elektrotechnik.\\

Die elektrische Kapazität zwischen zwei voneinander isolierten elektrisch leitenden Körpern ist gleich dem Verhältnis der Ladungsmenge Q, die auf diesen Leitern gespeichert ist ( + Q+ auf dem einen und Q- auf dem anderen), und der zwischen ihnen herrschenden elektrischen Spannung U:
$$C=\frac{Q}{U}$$
Sie wird dabei festgelegt durch die Dielektrizitätskonstante des isolierenden Mediums sowie die Geometrie der Körper, dazu zählt auch der Abstand. Dadurch stehen (sofern die Kapazität konstant ist) Q und U zueinander in einer proportionalen Beziehung.\\

Bei Akkumulatoren sowie Batterien benutzt man den Begriff „Kapazität“ für die maximale Ladungsmenge Q, welche in ihnen gespeichert werden kann. Sie wird in Amperestunden (Ah) angegeben. Diese Kapazität der elektrischen Ladung hat jedoch weder etwas mit der hier dargestellten elektrischen Kapazität (Farad) noch mit der Leistungskapazität (Watt) zu tun.\\

Bei einem Kondensator gilt allgemein (weil ...)
$$Q=C\cdot U$$
Nach dem Parallelschalten der beiden Kondensatoren kommt es zu einem Spannungsausgleich. Somit ist die neue Spannung $u'$ der beiden Kondensatoren konstant. Durch Umformen der obrigen Gleichung gilt also
$$\frac{Q_1}{C_u} = \frac{Q_2}{C_b} = u'$$
asdasdasd
$$ Q_{1} + Q_{2} = Q_{3} \Rightarrow m_{\rm{Eis}} \cdot L_f + m_{\rm{Eis}} \cdot c_{\rm{H_2O}} \cdot \Delta\vartheta_1 = m_{\rm{H_2o}} \cdot c_{\rm{H_2o}} \cdot \Delta\vartheta_2$$ $$ \Rightarrow L_f = \frac{m_{\rm{H2O}} \cdot c_{\rm{H_2o}} \cdot \Delta\vartheta_2 - m_{\rm{Eis}} \cdot c_{\rm{H_2O}} \cdot \Delta\vartheta_1}{m_{\rm{Eis}}}$$

NEUE ERKLÄRUNG DER FORMELN HIER

$$\frac{Q_1}{C_u} = \frac{Q_2}{C_b} = u'\quad \Rightarrow \quad Q_2 = u' \cdot C_b$$

$$Q_{Total} = C_b \cdot u$$

$$Q_{Total} = Q_1+Q_2 \quad \Rightarrow \quad Q_1=Q_{Total}-Q_2$$

$$C_u = \frac{Q_1}{u'} = \frac{Q_{Total}-Q_2}{u'} = \frac{C_b \cdot u - u' \cdot C_b}{u'}$$

\section{Versuchsaufbau}
\begin{figure}[H]

INSERT IMAGE HERE

\end{figure}

\section{Versuchsdurchführung}
\begin{table}[H]
    \centering
    \begin{tabular}{|c|c|c|}
        \hline
        \textbf{Beschreibung} & \textbf{Abkürzung} & \textbf{Wert} \\
        \hline
        Kapazität bekannter Kondensator & $C_{b}$ & $(4.38\pm 0.005 )\mu F$\\
        \hline
        Spannung bekannter Kondensator 1 & $u_{1}$ & $(260\pm 2.5)V$\\
        Spannung bekannter Kondensator 2 & $u_{2}$ & $(300\pm 2.5)V$\\
        Spannung bekannter Kondensator 3 & $u_{3}$ & $(280\pm 2.5)V$\\
        \hline
        Spannung nach dem Ausgleich 1 & $u_{1}'$ & $(178\pm 2.5)V$\\
        Spannung nach dem Ausgleich 2 & $u_{2}'$ & $(201\pm 2.5)V$\\
        Spannung nach dem Ausgleich 3 & $u_{3}'$ & $(192\pm 2.5)V$\\
        \hline
    \end{tabular}
\end{table}

INSERT DESCRIPTION HERE\\

Die im letzten Paragraphen genannten Schritte werden mit anderer Ausgangspannung wiederholt durchgeführt.
\section{Versuchsauswertung}

$$C_u = \frac{C_b \cdot u - u' \cdot C_b}{u'}$$

\subsection{Durchführung 1}

$C_{u_{max}} = \displaystyle{\frac{C_{b_{max}}\cdot u_{1_{max}}-u_{1_{min}}'\cdot C_{b_{min}}}{u_{1_{min}}'}}$\\\\

$C_{u_{min}} = \displaystyle{\frac{C_{b_{min}}\cdot u_{1_{min}}-u_{1_{max}}'\cdot C_{b_{max}}}{u_{1_{max}}'}}$\\\\

$\Rightarrow C_{u_1}=\displaystyle{\frac{C_{u_{max}}+C_{u_{min}}}{2}}$

\subsection{Durchführung 2}

$C_{u_{max}} = \displaystyle{\frac{C_{b_{max}}\cdot u_{2_{max}}-u_{2_{min}}'\cdot C_{b_{min}}}{u_{2_{min}}'}}$\\\\

$C_{u_{min}} = \displaystyle{\frac{C_{b_{min}}\cdot u_{2_{min}}-u_{2_{max}}'\cdot C_{b_{max}}}{u_{2_{max}}'}}$\\\\

$\Rightarrow C_{u_2}=\displaystyle{\frac{C_{u_{max}}+C_{u_{min}}}{2}}$

\subsection{Durchführung 2}

$C_{u_{max}} = \displaystyle{\frac{C_{b_{max}}\cdot u_{3_{max}}-u_{3_{min}}'\cdot C_{b_{min}}}{u_{3_{min}}'}}$\\\\

$C_{u_{min}} = \displaystyle{\frac{C_{b_{min}}\cdot u_{3_{min}}-u_{3_{max}}'\cdot C_{b_{max}}}{u_{3_{max}}'}}$\\\\

$\Rightarrow C_{u_3}=\displaystyle{\frac{C_{u_{max}}+C_{u_{min}}}{2}}$

\section{Kommentar / Diskussion}

\subsection{Genauigkeit}
Bei den beiden Versuchsdurchgängen wurden beim ersten Mal eine Abweichung von \textit{13\%} und beim zweiten Mal \textit{37\%} festgestellt.\\

Aufgrund der vielen systematischen Fehler, da nicht in einem abgschlossenen System experimentiert werden konnte, kann die Ungenauigkeit der Messresultate erklärt werden. Der Tabellenwert $\num{3.338 e5}\si{\J\per\kg}$ \cite{formelsammlung} wurde wie erwartet unterschritten, da einige Energie aus unserem System an die Umgebung verloren ging.\\

Es ist noch anzumerken, dass bei der Berechnung keine Fehlerschranke bei der Masse gemacht wurde. Dies ist zu begründen, dass diese Ungenauigkeit im Vergleich zur Temperaturmessung vernachlässigbar ist.

\subsection{Fehlerquellen}

Ein systematisch Fehler bestand darin, dass das Kalorimeter nicht zu 100\% isoliert und durch die Wände konstant Energie an die Umwelt abgegeben wird. Vorallem da das Kalorimeter nach oben offen war, entstanden dabei beträchtlich mehr Wärmeverluste am Wasser an die Umgebung als nur den Wänden.\\

Ein weiterer Fehler bestand darin, dass das Eis nicht vollständig mit dem Papier abgetrocknet werden konnte.\\

Beim der zweiten Versuchsdurchführung ist ein kleiner Fehler unterlaufen: Das Eis ist auf den Tisch gefallen und wurde dann mit dem Händen in das Kalorimeter befördert. Dabei ist ein Teil des Eises geschmolzen, weil Wärmeenergie von den Händen an das Eis abgegeben wurde. Somit ist die höhere Abweichung vom Tabellenwert im Vergleich zum ersten Versuchsdurchlauf begründet.

\end{document}
